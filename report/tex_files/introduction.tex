\section{Introducción}
Un papiloma es un tumor epitelial benigno que crece de manera exofítica \cite{Kozomara2007}, es decir, proyectándose hacia afuera en forma de proyecciones y de manera no agresiva ni propagándose por todo el cuerpo. En este contexto, "papila" se refiere a la proyección creada por el tumor, no a un tumor en una papila ya existente.
Cuando se utiliza sin contexto específico, con frecuencia se refiere a infecciones causadas por el virus del papiloma humano (VPH). Existen casi 200 tipos distintos de VPH \cite{Ljubojevic2014}, y muchos de ellos son carcinogénicos. Sin embargo, también existen otras condiciones que pueden causar papilomas, así como muchos casos en los que la causa no se conoce.
Las infecciones por el VPH de riesgo alto en ocasiones causan cáncer en las partes del cuerpo en las que el VPH infecta a las células. Por ejemplo, cáncer de cuello uterino, cáncer vulvar, cáncer vaginal, cáncer de pene, cáncer anal y cánceres orofaríngeos positivos para el VPH. La carga de los cánceres relacionados con el virus del papiloma humano causa cerca del 5\% \cite{papiloma} de todos los cánceres en el mundo, se calcula que 570.000 mujeres y 60.000 hombres tienen un cáncer relacionado con el VPH cada año \cite{papiloma} . El cáncer de cuello uterino es el más frecuente de todos los causados por el VPH, debido a que este es la causa de casi todos los cánceres de cuello uterino del mundo \cite{papiloma}.
