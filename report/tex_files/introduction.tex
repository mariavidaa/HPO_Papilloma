\section{Introducción}
Un papiloma es un tumor epitelial benigno que crece de manera exofítica \cite{Kozomara2007}, es decir, proyectándose hacia afuera en forma de proyecciones y de manera no agresiva ni propagándose por todo el cuerpo. En este contexto, "papila" se refiere a la proyección creada por el tumor, no a un tumor en una papila ya existente.
Cuando se utiliza sin contexto específico, con frecuencia se refiere a infecciones causadas por el virus del papiloma humano (VPH). Existen casi 200 tipos distintos de VPH \cite{Ljubojevic2014}, y muchos de ellos son carcinogénicos. Sin embargo, también existen otras condiciones que pueden causar papilomas, así como muchos casos en los que la causa no se conoce.
Las infecciones por el VPH de riesgo alto en ocasiones causan cáncer en las partes del cuerpo en las que el VPH infecta a las células. Por ejemplo, cáncer de cuello uterino, cáncer vulvar, cáncer vaginal, cáncer de pene, cáncer anal y cánceres orofaríngeos positivos para el VPH. La carga de los cánceres relacionados con el virus del papiloma humano causa cerca del 5\% \cite{papiloma} de todos los cánceres en el mundo, se calcula que 570.000 mujeres y 60.000 hombres tienen un cáncer relacionado con el VPH cada año \cite{papiloma} . El cáncer de cuello uterino es el más frecuente de todos los causados por el VPH, debido a que este es la causa de casi todos los cánceres de cuello uterino del mundo \cite{papiloma}.

\vspace{5pt}

\subsection{Información sobre los genes a estudiar} 
A continuación, veremos una breve información de los genes con mayor grado de interconexión que hemos encontrado al realizar un análisis fenotípico (HPO).

\vspace{3pt}

\textbf{AKT1}: La proteína kinasa B (AkT1) tiene un papel fundamental en el crecimiento y la supervivencia celular al transducir señales en la cascada de señalización celular (PI3K)/AKT, la cual genera mensajeros que participan en la regulación de la progresión del ciclo celular, adhesión y migración. La vía  PI3K/AKT es una de las que suelen estar afectadas en distintos tipos de cáncer en humanos, como el cáncer de ovario, de mama, y de cowden. Además, se asocia los niveles altos de fosforilación de la proteína con los peores pronósticos de cáncer \cite{Siegel2012}.

\textbf{TP53}: TP53 (tumor protein p53) es un gen supresor de tumores involucrado en procesos biológicos fundamentales para la estabilidad genética. Las mutaciones en este gen han sido asociadas con un peor pronóstico para pacientes con carcinoma oral de células escamosas, dando lugar a un cáncer más agresivo al combinarse con el VPH \cite{McKenna2021}.

\textbf{HRAS}: Este gen participa en la regulación de la vía de la proteína quinasa activada por mitógenos (MAPK) y mediada por la proteína quinasa Raf. En los últimos años se han definido una serie de síndromes con mutaciones en genes implicados en esta vía Ras/PAPK, entre ellos el síndrome de Costellos. Este síndrome refleja características cutáneas distintivas, como papillomas \cite{Siegel2012}.

\textbf{SDHD}: Las mutaciones en este gen están asociadas con la formación de tumores, incluyendo el paraganglioma hereditario. La transmisión de la enfermedad ocurre casi exclusivamente a través del alelo paterno, lo que sugiere que este locus puede estar impreso maternalmente. Hay pseudogenes para este gen en los cromosomas 1, 2, 3, 7 y 18. Resultados de empalme alternativos en múltiples variantes de transcripción \cite{Hensen2004}.
