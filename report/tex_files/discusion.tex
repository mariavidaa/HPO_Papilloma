\section{Discusión}

El estudio presenta un análisis exhaustivo de la red de genes asociados al fenotipo Papiloma, abordando aspectos clave que van desde la identificación de genes hasta la exploración de relaciones con fenotipos patológicos. A continuación, se discuten las implicaciones y hallazgos más relevantes del estudio.

\vspace{3pt}

En el estudio de red de interacción entre genes (sección 3.1), se destaca la identificación de 51 genes vinculados al fenotipo Papiloma. La representación visual de la red de interacción ofrece una visión intuitiva de cómo estos genes están conectados entre sí.

\vspace{3pt}

El análisis de enriquecimiento funcional (sección 3.2) profundiza en la funcionalidad de los genes identificados y la participación común de estos en vías metabólicas del sistema inmune, afectadas por carcinomas y relacionadas con el síndrome de Costellos y Cowden.  En particular, los vínculos encontrados del gen HRAS con la enfermedad de costellos concuerda con los hallazgos de otros autores, que postulan como mutaciones de dicho gen, que regula vías de transducción, tienen una gran relación con la enfermedad \cite{Siegel2012}.  Nuestros hallazgos también confirman lo evidenciado por otros científicos, en cuanto a como el cancer de endometrio se ha reconocido como uno de los componentes del síndrome de Cowden debido a alteraciones en las líneas germinales de PTEN y SDHB-D \cite{Mahdi2015}.